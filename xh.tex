\documentclass{report}
\usepackage[utf8]{inputenc}
\usepackage{amsmath,amssymb,amsthm,pxfonts,listings,color}
\usepackage[colorlinks]{hyperref}
\definecolor{gray}{rgb}{0.6,0.6,0.6}

\usepackage{caption}
\DeclareCaptionFormat{listing}{\llap{\color{gray}#1\hspace{10pt}}\tt{}#3}
\captionsetup[lstlisting]{format=listing, singlelinecheck=false, margin=0pt, font={bf}}

\lstset{columns=fixed,basicstyle={\tt},numbers=left,firstnumber=auto,basewidth=0.5em,showstringspaces=false,numberstyle={\color{gray}\scriptsize}}

\newcommand{\Ref}[2]{\hyperref[#2]{#1 \ref*{#2}}}

\lstnewenvironment{asmcode}       {}{}
\lstnewenvironment{cppcode}       {\lstset{language=c++}}{}
\lstnewenvironment{javacode}      {\lstset{language=java}}{}
\lstnewenvironment{javascriptcode}{}{}
\lstnewenvironment{htmlcode}      {\lstset{language=html}}{}
\lstnewenvironment{perlcode}      {\lstset{language=perl}}{}
\lstnewenvironment{rubycode}      {\lstset{language=ruby}}{}
\lstnewenvironment{pythoncode}    {\lstset{language=python}}{}

\lstnewenvironment{resourcecode}{}{}

\newcommand{\bs}{\textbackslash}
\setcounter{tocdepth}{0}
\lstnewenvironment{xhcode}{}{}

\title{xh}
\author{Spencer Tipping}

\begin{document}
\maketitle{}
\tableofcontents{}


\part{Language reference}\label{part:language-reference}
\chapter{Introduction}\label{chp:introduction}
  As a programming language, xh gives you two fairly uncommon invariants:

\begin{enumerate}
\item{Every value is fully expressible as a string, and behaves as such.}
       \label{item:values-are-strings}
\item{Every computation can be expressed as a series of}
       string-transformation rules.
       \label{item:computation-is-transformation}
\end{enumerate}

  xh's string transformations are all about expansion, which corresponds
  roughly to the kind of interpolation found in shell script or TCL. Unlike
  those languages, however, xh string interpolation itself has invariants, some
  of which you can disable. The semantics of xh are all defined in terms of the
  string representations of values, though xh is at liberty to use any
  representation that convincingly maintains the illusion that your values
  function as strings.

\section{Examples}\label{sec:examples}
    In these examples, \verb|$| indicates the bash prompt and the outermost
    \verb|()| indicate the xh prompt (neither needs to be typed).

\begin{verbatim}
bash                              xh
$ echo hi                         (. hi)
$ foo=bar                         (def foo bar)
$ echo $foo                       (. @foo)
$ echo "$foo"                     (. $foo)
$ echo "$(eval $foo)"             (. !foo)
$ echo $(eval $foo)               (. @!foo)
$ find . -name '*.txt'            (find . -name '*.txt')
$ ls name\ with\ spaces           (ls name\ with\ spaces)
$ rm x && touch x                 (rm x && touch x)
$ ls | wc -l                      (ls | wc -l)
$ cat foo > bar                   (cat foo > bar)

$ for f in $files; do             (map fn(rm $_ && touch $_) $files)
>   rm "$f" && touch "$f"
> done

$ if [[ -x foo ]]; then           (if (-x foo) (./foo arg1 arg2 @_))
>   ./foo arg1 arg2 "$@"
> fi

$ ls | while read f; do           (ls | filter -S)
>   [[ -S $f ]] && echo $f
> done
\end{verbatim}

    \noindent Some xh features have no analog in bash, for instance data
    structures:

\begin{verbatim}
clojure                           xh
(def m {})                        (def m {})
(assoc m :foo 5)                  {foo 5 @m}
(assoc m :foo 5)                  (assoc $m foo 5)
(dissoc m :foo :bar)              (dissoc $m foo bar)
(:foo m)                          ($m foo)
(get m :foo 0)                    ($m foo 0)
(map? m)                          (map? $m)
(contains? m :foo)                (contains? $m foo)

(def v [])                        (def v [])
(conj v 1 2 3)                    [@v 1 2 3]
(conj v 1 2 3)                    (push $v 1 2 3)

(def s #{})                       (def s s[])
(contains? s :foo)                ($s foo)
(contains? s :foo)                (contains? $s foo)

(fn [x] (inc x))                  fn(inc $_)
(fn [x] (inc x))                  (fn [$x] (inc $x))
(fn ([x]   (inc x))               (fn [$x]    (inc $x)
    ([x y] (+ x y)))                  [$x $y] (+ $x $y))
(comp f g h)                      (comp f g h)
(partial f x)                     (partial f x)
\end{verbatim}

\part{Self-hosting implementation}\label{part:self-hosting-implementation}
\chapter{xh-script parser}\label{chp:xh-script-parser}
  xh provides nestable regular expressions with value extraction, which means
  you can use them to write parsing expression grammars. It isn't as powerful
  as Perl 6, which also supports assertions and backtracking annotations, but
  it does let you invert the regexp match to produce the original (or a
  modified) string again.

\lstset{caption={modules/parse.xh},name={modules/parse.xh}}\begin{xhcode}
# Bidirectional quote/unquote relation
(def xh-parser
     (grammar rx" (?\$before $xh-ignored )
                  (?\$v $xh-vector | $xh-list | $xh-map | $xh-atom )
                  (?\$after $xh-ignored ) "

       xh-identifier rx" [^\s()\[\]{}$@\"']+ "
       xh-comment    rx" #(.*) "
       xh-ignored    rx" ($xh-comment | \s*)* "
       xh-atom       rx" $xh-hardstring | $xh-softstring
                       | $xh-bareword
                       | $xh-single-interpolation
                       | $xh-flat-interpolation "

       xh-hardstring rx" (?\$prefix $xh-identifier ?)
                         ' (?\$v ([^'\\]* | \\. )*) ' "

       xh-softstring rx" (?\$prefix $xh-identifier ?)
                         \" (?\$v [^\\\"\$\@]* | $xh-single-interpolation
                                               | $xh-flat-interpolation
                                               | $xh-interpolated-list
                                               | \\. ) \" "

       xh-bareword             rx" $xh-identifier "
       xh-single-interpolation rx" \$ ($xh-identifier | $xh-list) "
       xh-flat-interpolation   rx" \@ ($xh-identifier | $xh-list) "
       xh-vector               (xh-braced '[' ']')
       xh-list                 (xh-braced '(' ')')
       xh-map                  (xh-braced '{' '}')
       ^where
       (xh-braced $open $close) rx" (?\$prefix $xh-identifier ?)
                                    $open (?\$xs $xh-parser *) $close "))

# Evaluation semantics
(def (unquote (inv xh-parser $x)) (eval $x)
     ^where
     (eval xh-bareword[$x])    xh-string-val[$x]
     (eval xh-hardstring[@xs]) xh-string-val[(join "" (map eval $xs))]
     (eval xh-softstring[@xs]) xh-string-val[(join "" (map eval $xs))]

     (eval xh-vector{prefix $p xs $xs})
     xh-seq{prefix $p
            type   vector
            xs     (flatmap eval-single $xs)}

     (eval xh-map{prefix $p xs $xs})
     xh-seq{prefix $p
            type   map
            xs     (flatmap eval-single $xs)}

     (eval xh-list{prefix fn xs $xs}) xh-fn[$xs]
     # TODO
     ) \end{xhcode}

\part{Bootstrap implementation}\label{part:bootstrap-implementation}
\chapter{Self-replication}\label{chp:self-replication}
  {\bf Note:} This implementation requires Perl 5.14 or later, but the
  self-compiled xh image will run on anything back to 5.10. For this and other
  reasons, mostly performance-related, you should always use the xh-compiled
  image rather than bootstrapping in production.

\lstset{caption={boot/xh-header},name={boot/xh-header}}\begin{perlcode}
#!/usr/bin/env perl
BEGIN {eval(our $xh_bootstrap = q{
# xh: the X shell | https://github.com/spencertipping/xh
# Copyright (C) 2014, Spencer Tipping
# Licensed under the terms of the MIT source code license

# For the benefit of HTML viewers (long story):
# <body style='display:none'>
# <script src='http://spencertipping.com/xh/page.js'></script>
use 5.010;
package xh;
our %modules;
our @module_ordering;
our %eval_numbers = (1 => '$xh_bootstrap');

sub with_eval_rewriting(&) {
  my @result = eval {$_[0]->(@_[1..$#_])};
  $@ =~ s/\(eval (\d+)\)/$eval_numbers{$1}/eg if $@;
  die $@ if $@;
  @result;
}

sub named_eval {
  my ($name, $code) = @_;
  $eval_numbers{$1 + 1} = $name if eval('__FILE__') =~ /\(eval (\d+)\)/;
  with_eval_rewriting {eval $code};
}

our %compilers = (pl => sub {
  my $package = $_[0] =~ s/\./::/gr;
  named_eval $_[0], "{package ::$package;\n$_[1]\n}";
  die "error compiling module $_[0]: $@" if $@;
});

sub defmodule {
  my ($name, $code, @args) = @_;
  chomp($modules{$name} = $code);
  push @module_ordering, $name;
  my ($base, $extension) = split /\.(\w+$)/, $name;
  die "undefined module extension '$extension' for $name"
    unless exists $compilers{$extension};
  $compilers{$extension}->($base, $code, @args);
}

chomp($modules{bootstrap} = $::xh_bootstrap);
undef $::xh_bootstrap; \end{perlcode}

  At this point we need a way to reproduce the image. Since the bootstrap code
  is already stored, we can just wrap it and each defined module into an
  appropriate \verb|BEGIN| block.

\lstset{caption={boot/xh-header (continued)},name={boot/xh-header}}\begin{perlcode}
sub image {
  my @pieces = "#!/usr/bin/env perl";
  push @pieces, "BEGIN {eval(our \$xh_bootstrap = <<'_')}",
                $modules{bootstrap},
                '_';
  push @pieces, "BEGIN {xh::defmodule('$_', <<'_')}",
                $modules{$_},
                '_' for @module_ordering;
  push @pieces, "xh::main::main;\n__DATA__";
  join "\n", @pieces;
}
})} \end{perlcode}

\end{document}
