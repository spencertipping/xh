\documentclass{report}
\usepackage[utf8]{inputenc}
\usepackage{amsmath,amssymb,amsthm,pxfonts,listings,color}
\usepackage[colorlinks]{hyperref}
\definecolor{gray}{rgb}{0.6,0.6,0.6}

\usepackage{caption}
\DeclareCaptionFormat{listing}{\llap{\color{gray}#1\hspace{10pt}}\tt{}#3}
\captionsetup[lstlisting]{format=listing, singlelinecheck=false, margin=0pt, font={bf}}

\lstset{columns=fixed,basicstyle={\tt},numbers=left,firstnumber=auto,basewidth=0.5em,showstringspaces=false,numberstyle={\color{gray}\scriptsize}}

\newcommand{\Ref}[2]{\hyperref[#2]{#1 \ref*{#2}}}

\lstnewenvironment{asmcode}       {}{}
\lstnewenvironment{cppcode}       {\lstset{language=c++}}{}
\lstnewenvironment{javacode}      {\lstset{language=java}}{}
\lstnewenvironment{javascriptcode}{}{}
\lstnewenvironment{htmlcode}      {\lstset{language=html}}{}
\lstnewenvironment{perlcode}      {\lstset{language=perl}}{}
\lstnewenvironment{rubycode}      {\lstset{language=ruby}}{}
\lstnewenvironment{pythoncode}    {\lstset{language=python}}{}

\lstnewenvironment{resourcecode}{}{}

\newcommand{\bs}{\textbackslash}
\setcounter{tocdepth}{0}
\lstnewenvironment{xhcode}{}{}

\title{X shell}
\author{Spencer Tipping}

\begin{document}
\maketitle{}
\tableofcontents{}


\part{Language reference}\label{part:language-reference}
\chapter{Similarities to TCL}\label{chp:similarities-to-tcl}
  Every xh value is a string. This includes functions, closures, lazy
  expressions, scope chains, call stacks, and heaps. Asserting string
  equivalence makes it possible to serialize any value losslessly, including a
  running xh process.\footnote{Note that things like active socket connections
  and external processes will be proxied, however; xh can't migrate
  system-native things.}

  Although the string equivalence is available, most operations have
  higher-level structure. For example, the \verb|$| operator, which performs
  string interpolation, interpolates values in such a way that two things are
  true:

\begin{enumerate}
\item{No interpolated value will be further interpolated (idempotence).}
\item{The interpolated value will be read as a single list element.}
\end{enumerate}

  For example:

\begin{verbatim}
$ def bar bif
$ def foo "hi there \$bar!"
$ def baz $foo                      # no quoting necessary here by (2)
$ echo $baz
hi there $bar!                      # $bar unevaluated by (1)
$
\end{verbatim}

  This interpolation structure can be overridden by using one of three
  alternative forms of \verb|$|:

\begin{verbatim}
$ def bar bif
$ def foo "hi there \$bar!"
$ echo $!foo                        # allow re-interpolation
hi there bif!
$ count [$foo]                      # single element
1
$ count [$@foo]                     # multiple elements
3
$ nth [$@!foo] 2                    # multiple and re-interpolation
bif!
$
\end{verbatim}

  All string values in xh programs are lifted into reader-safe quotations. This
  causes any ``active'' characters such as \verb|$| to be prefixed with
  backslashes, a transformation you can mostly undo by using \verb|$@!|. The
  only thing you can't undo is bracket balancing, which if undone would wreak
  havoc on your programs. You can see the effect of balancing by doing
  something like this:

\begin{verbatim}
$ def foo "[[[["
$ def bar [$@!foo]
$ echo $bar
[\[\[\[\[]
$
\end{verbatim}

  We can't get xh to create an unbalanced list through any series of rewriting
  operations, since the contract is that any active list characters are either
  positive and balanced, or escaped.

\chapter{Similarities to Lisp}\label{chp:similarities-to-lisp}
  xh is strongly based on the Lisp family of languages, most visibly in its
  homoiconicity.

\part{Bootstrap implementation}\label{part:bootstrap-implementation}
\chapter{Self-replication}\label{chp:self-replication}
\lstset{caption={boot/xh-header},name={boot/xh-header}}\begin{perlcode}
#!/usr/bin/env perl
BEGIN {
print STDERR q{
NOTE: Development image

If you see this note after installing the shell, it's probably because
you're running a version that has not yet rebuilt itself (maybe you got the
wrong file from the Git repo?). You can do this, but it will be really
slow and may use a lot of memory. There are two ways to fix this:

1. Download the standard image from http://spencertipping.com/xh
2. Have this image recompile itself by running xh.recompile-in-place (this
   will take some time because it stress-tests your Perl runtime)

Note also that bootstrapping requires Perl 5.14 or later, whereas running a
compiled image just requires Perl 5.10.

};
}

BEGIN {eval(our $xh_bootstrap = q{
# xh: the X shell | https://github.com/spencertipping/xh
# Copyright (C) 2014, Spencer Tipping
# Licensed under the terms of the MIT source code license

# For the benefit of HTML viewers (long story):
# <body style='display:none'>
# <script src='http://spencertipping.com/xh/page.js'></script>
use 5.014;
package xh;
our %modules;
our @module_ordering;

our %compilers = (pl => sub {
  my $package = $_[0] =~ s/\./::/gr;
  eval "{package ::$package;\n$_[1]\n}";
  die "error compiling module $_[0]: $@" if $@;
});

sub defmodule {
  my ($name, $code, @args) = @_;
  chomp($modules{$name} = $code);
  push @module_ordering, $name;
  my ($base, $extension) = split /\.(\w+$)/, $name;
  die "undefined module extension '$extension' for $name"
    unless exists $compilers{$extension};
  $compilers{$extension}->($base, $code, @args);
}

chomp($modules{bootstrap} = $::xh_bootstrap);
undef $::xh_bootstrap; \end{perlcode}

  At this point we need a way to reproduce the image. Since the bootstrap code
  is already stored, we can just wrap it and each defined module into an
  appropriate \verb|BEGIN| block.

\lstset{caption={boot/xh-header (continued)},name={boot/xh-header}}\begin{perlcode}
sub image {
  my @pieces = "#!/usr/bin/env perl";
  push @pieces, "BEGIN {eval(our \$xh_bootstrap = <<'_')}",
                $modules{bootstrap},
                '_';
  push @pieces, "BEGIN {xh::defmodule('$_', <<'_')}",
                $modules{$_},
                '_' for @module_ordering;
  push @pieces, "xh::main::main;\n__DATA__";
  join "\n", @pieces;
}
})} \end{perlcode}

\end{document}
